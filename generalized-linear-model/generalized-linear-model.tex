\chapter{Generalized Linear Model}

\section{Exponential Family}

\begin{definition}[Exponential Family] \label{def:exponential-family}
    An exponential family of probability distributions as those distributions whose density is defined to be
    \begin{equation}
        f\left(y\mid\theta,\phi\right)=\exp\left[\frac{y\theta-b(\theta)}{a(\phi)}+c(y,\phi)\right]
    \end{equation}
\end{definition}

\begin{property}
    The exponential family have the following properties,
    \begin{equation*}
        E(Y)=b^{\prime}(\theta)\quad\operatorname{Var}(Y)=b^{\prime\prime}(\theta)a(\phi).
    \end{equation*}
\end{property}

\begin{proof}

\end{proof}

\begin{landscape}
    \begin{table}[hpt]
        \centering
        \caption{Common Distributions of Exponential Family}
        \begin{tabular}{ccccccccc}
            \toprule
            Distribution & Parameter(s)      & $\theta$                         & $\phi$       & $b(\theta)$                     & $a(\phi)$ & $c(y,\phi)$                                                   & $E(Y)$                            & $\operatorname{Var}(Y)$                            \\
            \midrule
            Normal       & $N(\mu,\sigma^2)$ & $\mu$                            & $\sigma^{2}$ & $\frac{\theta^{2}}{2}$          & $\phi$    & $-\frac{1}{2}\left[\frac{y^{2}}{\phi}+\log (2\pi\phi)\right]$ & $\theta$                          & $\phi$                                             \\
            Bernoulli    & $\text{Bern}(p)$  & $\log\left(\frac{p}{1-p}\right)$ & $1$          & $\log\left(1+e^{\theta}\right)$ & $1$       & $0$                                                           & $\frac{e^{\theta}}{1+e^{\theta}}$ & $\frac{e^{\theta}}{\left(1+e^{\theta}\right)^{2}}$ \\
            Poisson      & $P(\mu)$          & $\log(\mu)$                      & $1$          & $\mathrm{e}^{\theta}$           & $1$       & $-\log(y!)$                                                   & $\mathrm{e}^{\theta}$             & $\mathrm{e}^{\theta}$                              \\
            \bottomrule
        \end{tabular}
    \end{table}
\end{landscape}

\section{Model Assumption}

Suppose the response $Y$ has a distribution in the exponential family
\begin{equation*}
    f\left(y\mid\theta,\phi\right)=\exp\left[\frac{y\theta-b(\theta)}{a(\phi)}+c(y,\phi)\right]
\end{equation*}
with link function $g$, such that,
\begin{equation}
    E\left(Y\mid\mathbf{X}\right)=\mu=g^{-1}(\eta),\quad\eta=\mathbf{X}^{\prime}\boldsymbol{\beta}
\end{equation}
where the link function provides the relationship between the linear predictor and the mean of the distribution function. If $\eta=\theta$, the link function is called \textbf{canonical link function}.

\begin{remark}
    A generalized linear model (GLM) is a flexible generalization of ordinary linear regression that allows for the response variable to have an error distribution other than the normal distribution.
\end{remark}

\begin{table}[hpt]
    \centering
    \caption{Commonly Used Link Functions}
    \begin{tabular}{clcc}
        \toprule
        Distribution & Support of Distribution  & Link Function $g(\mu)$               & Mean Function $g^{-1}(\eta)$ \\
        \midrule
        Normal       & real:$(-\infty,+\infty)$ & $\mu$                                & $\eta$                       \\
        Bernoulli    & integer: $\{0,1\}$       & $\log\left(\frac{\mu}{1-\mu}\right)$ & $\frac{1}{1+\exp(-\eta)}$    \\
        Poisson      & integer: $0,1,2,\ldots$  & $\log\left(\mu\right)$               & $\exp\left(\eta\right)$      \\
        \bottomrule
    \end{tabular}
\end{table}

\section{Model Estimation}

\subsection{Maximum Likelihood}

Suppose the log-likelihood function be
\begin{equation}
    \ell\left(\boldsymbol{\beta}\mid\mathbf{X},y\right)=\log\left[f\left(y\mid\theta,\phi\right)\right]=\log\left[f\left(y\mid g^{-1}(\eta),\phi\right)\right]
\end{equation}
where $g$ is the canonical link function and $\eta=\mathbf{X}^{\prime}\boldsymbol{\beta}$.

Let
\begin{equation*}
    U\left(\boldsymbol{\beta}\right)=\frac{\partial\ell\left(\boldsymbol{\beta}\right)}{\partial\boldsymbol{\beta}},\quad A\left(\boldsymbol{\beta}\right)=-\frac{\partial^{2}\ell\left(\boldsymbol{\beta}\right)}{\partial\boldsymbol{\beta}^{\prime}\partial\boldsymbol{\beta}}
\end{equation*}
be the score function and observed information matrix.

If $\hat{\boldsymbol{\beta}}$ is the maximum likelihood estimate, then
\begin{equation*}
    U\left(\hat{\boldsymbol{\beta}}\right)=\mathbf{0}
\end{equation*}

By mean value theorem,
\begin{equation*}
    \begin{aligned}
                    & U\left(\hat{\boldsymbol{\beta}}\right)-U\left(\boldsymbol{\beta}_{0}\right)=\frac{\partial U\left(\boldsymbol{\beta}^{*}\right)}{\partial\boldsymbol{\beta}}\left(\hat{\boldsymbol{\beta}}-\boldsymbol{\beta}_{0}\right) \\
        \Rightarrow & -U\left(\boldsymbol{\beta}_{0}\right)=-A\left(\boldsymbol{\beta}^{*}\right)\left(\hat{\boldsymbol{\beta}}-\boldsymbol{\beta}_{0}\right)
    \end{aligned}
\end{equation*}
where $\boldsymbol{\beta}^{*}\in\left[\boldsymbol{\beta}_{0},\hat{\boldsymbol{\beta}}\right]$. Thus,
\begin{equation*}
    \hat{\boldsymbol{\beta}}=\boldsymbol{\beta}_{0}+A^{-1}\left(\boldsymbol{\beta}^{*}\right)U\left(\boldsymbol{\beta}_{0}\right)
\end{equation*}

Suppose $\hat{\boldsymbol{\beta}}_{t},\hat{\boldsymbol{\beta}}_{t+1}$ be the maximum likelihood estimate at the t-th and (t+1)-th iterations, respectively. Two algorithms can be used to obtain the maximum likelihood estimate $\hat{\boldsymbol{\beta}}$.

\begin{enumerate}
    \item Newton-Raphson Method:
          \begin{equation}
              \hat{\boldsymbol{\beta}}_{t+1}=\hat{\boldsymbol{\beta}}_{t}+A^{-1}\left(\hat{\boldsymbol{\beta}}_{t}\right)U\left(\hat{\boldsymbol{\beta}}_{t}\right)\Leftrightarrow A\left(\hat{\boldsymbol{\beta}}_{t}\right)\hat{\boldsymbol{\beta}}_{t+1}=A\left(\hat{\boldsymbol{\beta}}_{t}\right)\hat{\boldsymbol{\beta}}_{t}+U\left(\hat{\boldsymbol{\beta}}_{t}\right)
          \end{equation}
          where
          \begin{equation}
              U\left(\boldsymbol{\beta}\right)=\frac{\partial\ell\left(\boldsymbol{\beta}\right)}{\partial\boldsymbol{\beta}}
          \end{equation}
          is the score function and
          \begin{equation}
              A\left(\boldsymbol{\beta}\right)=-\frac{\partial^{2}\ell\left(\boldsymbol{\beta}\right)}{\partial\boldsymbol{\beta}^{\prime}\partial\boldsymbol{\beta}}
          \end{equation}
          is the observed information matrix.
    \item Fisher’s Scoring Method:
          \begin{equation}
              \hat{\boldsymbol{\beta}}_{t+1}=\hat{\boldsymbol{\beta}}_{t}+I^{-1}\left(\hat{\boldsymbol{\beta}}_{t}\right)U\left(\hat{\boldsymbol{\beta}}_{t}\right)\Leftrightarrow I\left(\hat{\boldsymbol{\beta}}_{t}\right)\hat{\boldsymbol{\beta}}_{t+1}=I\left(\hat{\boldsymbol{\beta}}_{t}\right)\hat{\boldsymbol{\beta}}_{t}+U\left(\hat{\boldsymbol{\beta}}_{t}\right)
          \end{equation}
          where $U\left(\boldsymbol{\beta}\right)$ is the score function and
          \begin{equation}
              I\left(\boldsymbol{\beta}\right)=E\left[A\left(\boldsymbol{\beta}\right)\right]=-E\left[\frac{\partial^{2}\ell\left(\boldsymbol{\beta}\right)}{\partial\boldsymbol{\beta}^{\prime}\partial\boldsymbol{\beta}}\right]
          \end{equation}
          is the Fisher information matrix.
\end{enumerate}

\subsection{Bayesian Methods}
