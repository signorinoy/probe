\chapter{Convex Sets}

\section{Affine and Convex Sets}

\subsection{Affine Sets}
A nonempty set \(C\) is said to be \textbf{affine set}, if
\begin{equation*}
	\forall x_1,x_2\in C, \theta\in\bbR, \theta x_1+(1-\theta)x_2\in C.
\end{equation*}
\begin{definition}[Affine Set]

\end{definition}

\subsection{Convex Sets}

\begin{definition}[Convex Set]
	A nonempty set \(C\) is said to be \textbf{convex set}, if
	\begin{equation*}
		\forall x_1,x_2\in C,\theta\in[0,1], \theta x_1+(1-\theta)x_2\in C.
	\end{equation*}
\end{definition}

\begin{definition}[Convex Hull]
	The \textbf{convex hull} of said to be set \(C\), denoted by \(\text{conv } C\) is a set of all convex combinations of points in \(C\),
	\begin{equation*}
		\text{conv } C=\{\theta_1x_1+\cdots+\theta_{k}x_{k}|x_i\in C;\theta_i\geq 0,i=1,\ldots,k;\theta_1+\cdots+\theta_k=1\}.
	\end{equation*}
\end{definition}

\begin{remark}
	The convex hull \(\text{conv } C\) is always convex, which is the minimal convex set that contains \(C\).
\end{remark}

\subsection{Cones}

\begin{definition}[Cone]
	A nonempty set \(C\) is said to be \textbf{cone}, if
	\begin{equation*}
		\forall x\in C,\theta\geq 0,\theta x\in C.
	\end{equation*}
\end{definition}

\begin{definition}[Convex Cone]
	A nonempty set \(C\) is said to be \textbf{convex cone}, if
	\begin{equation*}
		\forall x_1,x_2\in C, \theta_1,\theta_2\geq 0,\theta_1 x_1+\theta_2 x_2\in C.
	\end{equation*}
\end{definition}

\section{Some Important Examples}

\begin{definition}[Hyperplane]
	A hyperplane is defined to be \(\{x|a^{\top}x=b\}\), where \(a\in\bbR^n,a\neq 0,b\in\bbR\).
\end{definition}

\begin{definition}[Halfspace]
	A hyperplane is defined to be \(\{x|a^{\top}x\leq b\}\), where \(a\in\bbR^n,a\neq 0,b\in\bbR\).
\end{definition}

\begin{definition}[(Euclidean) Ball]
	A (Euclidean) ball in \(\bbR^n\) with center \(x_c\) and radius \(r\) is defined to be
	\begin{equation*}
		B(x_c,r)=\{x\vert\|x-x_c\|_2\leq r\}=\{x_c+ru\vert\|u\|_2\leq 1\},
	\end{equation*}
	where \(r>0\).
\end{definition}

\begin{definition}[Ellipsoid]
	A Ellipsoid in \(\bbR^n\) with center \(x_c\) is defined to be
	\begin{equation*}
		\mathcal{E}=\{x\vert(x-x_c)^{\top}P^{-1}(x-x_c)\leq 1\}=\{x_c+Au\vert \|u_2\|\leq 1\},
	\end{equation*}
	where \(P\in\bbS^{n}_{++}\) (symmetric positive definite).
\end{definition}

\section{Generalized Inequalities}

\subsection{Definition of Generalized Inequalities}

\begin{definition}[Proper Cone]\label{def:proper-cone}
	A cone \(K\subseteq\bbR^n\) is said to be a proper cone, if
	\begin{itemize}
		\item \(K\) is convex.
		\item \(K\) is closed.
		\item \(K\) is solid (nonempty interior).
		\item \(K\) is pointed (contains no line).
	\end{itemize}
\end{definition}

\begin{definition}[Generalized Inequalities]\label{def:generalized-inequalities}
	The partial ordering on \(\bbR^n\) defined by proper cone \(K\), if
	\begin{equation}
		y-x\in K,
	\end{equation}
	which can be denoted by
	\begin{equation}
		x\preceq_{K}y\ \text{or}\ y\succeq_{K}x.
	\end{equation}
	The strict partial ordering on \(\bbR^n\) defined by proper cone \(K\), if
	\begin{equation}
		y-x\in\operatorname{int}K,
	\end{equation}
	which can be denoted by
	\begin{equation}
		x\prec_{K}y\ \text{or}\ y\succ_{K}x.
	\end{equation}
\end{definition}

\begin{remark}
	When \(K=\bbR_{+}\), the partial ordering \(\preceq_{K}\) is the usual ordering \(\leq\) on \(\bbR\), and the strict partial ordering \(\prec_{K}\) is the usual strict ordering \(<\) on \(\bbR\).
\end{remark}

\subsection{Properties of Generalized Inequalities}

\begin{theorem}[Properties of Generalized Inequalities]
	A generalized inequality \(\preceq_{K}\) has the following properties:
	\begin{itemize}
		\item Preserved under addition:
		\item Transitive:
		\item Preserved under nonnegative  scaling:
		\item Reflexive:
		\item Antisymmetric:
		\item Preserved under limits:
	\end{itemize}
	A strict generalized inequality \(\prec_{K}\) has the following properties:
\end{theorem}
