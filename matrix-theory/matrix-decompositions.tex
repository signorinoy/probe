\chapter{Matrix Decompositions}\label{chapter:matrix-decompositions}

\section{Spectral Decomposition}

\begin{definition}[Eigenvectors and Eigenvalues]
	A (non-zero) vector \(\bfv\) of dimension \(n\) is an \textbf{eigenvector} of a square \(n\times n\) matrix \(\bfA\), if
	\begin{equation}
		\bfA\bfv=\lambda\bfv
	\end{equation}
	where \(\lambda\) is a scalar, termed the \textbf{eigenvalue} corresponding to \(\bfv\).
\end{definition}

\begin{definition}[Spectral Decomposition]
	For any \(n\times n\) matrix with \(n\) linearly independent eigenvectors \(q_{i},i=1,\ldots,n\). Then \(\bfA\) can be factorized as
	\begin{equation*}
		\bfA=\bfQ\bfLambda\bfQ^{-1}
	\end{equation*}
	where \(\bfQ\) is the square \(n\times n\) matrix whose \(i\)-th column is the eigenvector \(\bfq_{i}\) of \(\bfA\), and \(\bfLambda\) is the diagonal matrix whose diagonal elements are the corresponding eigenvalues, \(\bfLambda=\lambda_{i}\). This factorization is called eigendecomposition or sometimes spectral decomposition.
\end{definition}

\begin{example}[Real Symmetric Matrices]
	As a special case, for every \(n\times n\) real symmetric matrix, the eigenvalues are real and the eigenvectors can be chosen as real and orthonormal. Thus a real symmetric matrix \(\bfA\) can be decomposed as
	\begin{equation}
		\bfA=\bfQ\bfLambda\bfQ^{\prime}
	\end{equation}
	where \(\bfQ\) is an orthogonal matrix whose columns are  eigenvectors of \(\bfA\), and \(\bfLambda\) is a diagonal matrix whose entries are the eigenvalues of \(\bfA\).
\end{example}

\section{Singular Value Decomposition}

\begin{definition}[Singular Value Decomposition]
	For any matrix \(\bfA\in\bbR^{m\times n}\), we have
	\begin{equation}
		\bfA=\mathbf{U}\bfSigma\bfV^{\prime}
	\end{equation}
	where
	\begin{itemize}
		\item \(\mathbf{U}\in\bbR^{m\times m}\) is an orthogonal matrix whose columns are the eigenvectors of \(\bfA\bfA^{\prime}\)
		\item \(\bfV\in\bbR^{n\times n}\) is an orthogonal matrix whose columns are the eigenvectors of \(\bfA^{\prime}\bfA\)
		\item \(\bfSigma\in\bbR^{m\times n}\) is an all-zero matrix except for the first \(r\) diagonal elements
		      \begin{equation*}
			      \sigma_{i}=\bfSigma_{ii},\quad i=1,2,\ldots,r
		      \end{equation*}
		      which are called singular values, which are the square roots of the eigenvalues of \(\bfA^{\prime}\bfA\) and of \(\bfA\bfA^{\prime}\) (these two matrices have the same eigenvalues)
	\end{itemize}
\end{definition}

\begin{remark}
	We assume above that the singular values are sorted in descending order and the eigenvectors are sorted according to descending order of their eigenvalues.
\end{remark}

\begin{proof}
	Without loss of generality, we assume \(m\geq n\). Since for the case \(n>m\), can then be established by transposing the SVD of \(\bfA^{\prime}\),
	\begin{equation*}
		\bfA=\left(\bfA^{\prime}\right)^{\prime}=\left(\mathbf{U}^{\prime}\bfSigma\bfV\right)^{\prime}=\bfV^{\prime}\left(\mathbf{U}^{\prime}\bfSigma\right)^{\prime}=\bfV^{\prime}\bfSigma\mathbf{U}
	\end{equation*}

	For \(m\geq n\), suppose \(\operatorname{rank}(A)=r\), and then \(\operatorname{rank}\left(\bfA^{\prime}\bfA\right)=r\) and the spectral decomposition of \(\bfA^{\prime}\bfA\) be
	\begin{equation*}
		\bfA^{\prime}\bfA\bfV=\bfV\operatorname{diag}\left(\sigma_{1}^{2},\ldots,\sigma_{r}^{2},0,\ldots,0\right)
	\end{equation*}
	where \(\sigma_{i}^{2}\) are the eigenvalues of \(\bfA^{\prime}\bfA\) and the columns of \(\bfV\), denoted \(\boldsymbol{v}^{(i)}\), are the corresponding orthonormal eigenvectors.

	Let
	\begin{equation*}
		\bfu^{(i)}=\frac{\bfA\boldsymbol{v}^{(i)}}{\sigma_{i}}
	\end{equation*}
	then
	\begin{gather*}
		\bfA^{\prime}\bfu^{(i)}=\frac{\bfA^{\prime}\bfA\boldsymbol{v}^{(i)}}{\sigma_{i}}=\sigma_{i}\boldsymbol{v}^{(i)}\Rightarrow                       \\
		\bfA\bfA^{\prime}\bfu^{(i)}=\sigma_{i}\bfA\boldsymbol{v}^{(i)}=\sigma_{i}^{2}\bfu^{(i)}
	\end{gather*}
	implying that \(\bfu^{(i)}\) are eigenvectors of \(\bfA\bfA^{\prime}\) corresponding to eigenvalues \(\sigma_{i}^{2}\).

	Since the eigenvectors \(\boldsymbol{v}^{(i)}\) are orthonormal, then so are the eigenvectors \(\bfu^{(i)}\)
	\begin{equation*}
		\left(\bfu^{(i)}\right)^{\prime}\bfu^{(j)}=\frac{\left(\boldsymbol{v}^{(i)}\right)^{\prime}\bfA^{\prime}\bfA\boldsymbol{v}^{(j)}}{\sigma_{i}^{2}}=\left(\boldsymbol{v}^{(i)}\right)^{\prime}\boldsymbol{v}^{(j)}=\begin{cases}1 & i=j \\ 0 & i \neq j\end{cases}
	\end{equation*}

	We have thus far a matrix \(\bfV\) whose columns are eigenvectors of \(\bfA^{\prime}\bfA\) with eigenvalues \(\sigma_{i}^{2}\), and a matrix \(\mathbf{U}\) whose columns are \(r\) eigenvectors of \(\bfA\bfA^{\prime}\) corresponding to eigenvalues \(\sigma_{i}^{2}\).

	We extend the set of eigenvectors \(\bfu^{(i)},\,i=1,\ldots,r\) by adding orthonormal vectors \(\bfu^{(i)},\,i=r+1,\ldots,m\) that span the null space \(\operatorname{null}\left(\bfA\bfA^{\prime}\right)\). Thus, the collection \(\bfu^{(i)},\,i=1,\ldots,m\) forms a complete orthonormal basis of eigenvectors for \(\bfA\bfA^{\prime}\), where the corresponding eigenvalues are \(\sigma_{i}^{2}\) for \(i\leq r\) and zero for \(i>r\).

	Since
	\begin{equation*}
		\left[\mathbf{U}^{\prime}\bfA\bfV\right]_{ij}=\left(\bfu^{(i)}\right)^{\prime}\bfA\boldsymbol{v}^{(j)}=\begin{cases}\sigma_{j}\left(\bfu^{(i)}\right)^{\prime}\bfu^{(j)} & i \leq r \\ 0 & i>r\end{cases}
	\end{equation*}
	we get
	\begin{equation*}
		\mathbf{U}^{\prime}\bfA\bfV=\bfSigma
	\end{equation*}
	where
	\begin{equation*}
		\bfSigma=\begin{pmatrix}
			\operatorname{diag}\left(\sigma_{1},\ldots,\sigma_{n}\right) \\
			\bfzero
		\end{pmatrix}
		,\quad\sigma_{i}=0\forall r<i\leq n
	\end{equation*}

	Consequentially, we get the desired decompositions
	\begin{equation*}
		\bfA=\mathbf{U}\bfSigma\bfV^{\prime}
	\end{equation*}
\end{proof}

\subsection{Relationship to Matrix Norm}

\begin{theorem}
	For any matrix \(\bfA\in\bbR^{m\times n}\),
	\begin{equation}
		\|\bfA\|_{2}=\sigma_{\max}(\bfA)
	\end{equation}
\end{theorem}

\begin{proof}
	For any matrix \(\bfA\in\bbR^{m\times n}\), the SVD implies that,
	\begin{equation*}
		\|\bfA\|_{2}=\sup_{\bfx\neq 0}\frac{\|\bfA\bfx\|_{2}}{\|\bfx\|_{2}}=\sup_{\bfx\neq 0}\frac{\left\|\mathbf{U}\bfSigma\bfV^{\prime}\bfx\right\|_{2}}{\|\bfx\|_{2}}
	\end{equation*}

	Since \(\mathbf{U}\) is unitary, that is,
	\begin{equation*}
		\left\|\mathbf{U}\bfx\right\|_{2}^{2}=\bfx^{\top}\mathbf{U}^{\prime}\mathbf{U}\bfx^{T}\bfx=\left\|\bfx\right\|_{2}^{2},\quad\forall\bfx\in\bbR^{m}
	\end{equation*}
	thus,
	\begin{equation*}
		=\sup_{\bfx\neq 0}\frac{\left\|\bfSigma\bfV^{\prime}\bfx\right\|_{2}}{\|\bfx\|_{2}}
	\end{equation*}

	Let \(\mathbf{y}=\bfV^{\prime}\bfx\), and since \(\bfV\) is unitary, we have
	\begin{equation*}
		\|\mathbf{y}\|_{2}=\left\|\bfV^{\prime}\bfx\right\|_{2}=\|\bfx\|_{2}=1
	\end{equation*}
	thus,
	\begin{equation*}
		=\sup_{\mathbf{y}\neq 0}\frac{\|\bfSigma\mathbf{y}\|_{2}}{\|\bfV\mathbf{y}\|_{2}}=\sup_{\mathbf{y}\neq 0}\frac{\left(\sum_{i=1}^{r}\sigma_{i}^{2}\left|y_{i}\right|^{2}\right)^{\frac{1}{2}}}{\left(\sum_{i=1}^{r}\left|y_{i}\right|^{2}\right)^{\frac{1}{2}}}\leq\sigma_{\max}(\bfA)
	\end{equation*}
	which takes ``\(=\)'', if \(\boldsymbol{y}=\left(1,0,\ldots,0\right)^{\prime}\).
\end{proof}

\begin{theorem}
	For any matrix \(\bfA\in\bbR^{m\times n}\), suppose \(\operatorname{rank}(A)=n\), then
	\begin{equation}
		\min_{\|\bfx\|_{2}=1}\|\bfA\bfx\|_{2}=\sigma_{n}(\bfA)
	\end{equation}
\end{theorem}

\begin{proof}
	The proof process is analogous to the above theorem.
\end{proof}

\begin{remark}
	If \(\operatorname{rank}(\bfA)<n\), then there is an \(\bfx\) such that the minimum is zero.
\end{remark}
