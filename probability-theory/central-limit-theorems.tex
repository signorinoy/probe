\chapter{Central Limit Theorems}

% \begin{introduction}
%     \item Classic Central Limit Theorem
%     \item Central Limit Theorem for independent non-identical Random Variables
%     \item Central Limit Theorem for Dependent Random Variables
% \end{introduction}

\section{Central Limit Theorem}

\subsection{The De Moivre-Laplace Theorem}

\begin{lemma}[Stirling's Formula] \label{lem:stirling}
	\begin{equation}
		n ! \sim \sqrt{2 \pi} n^{n+\frac{1}{2}} e^{-n} \text{ as } n \rightarrow \infty.
	\end{equation}
\end{lemma}

\begin{proof}

\end{proof}

\begin{lemma} \label{lem:exp}
	If $c_j\rightarrow 0$, $a_j\rightarrow\infty$ and $a_jc_j\rightarrow\lambda$, then
	\begin{equation}
		\left(1+c_j\right)^{a_j}\rightarrow e^\lambda.
	\end{equation}
\end{lemma}

\begin{proof}

\end{proof}

\begin{theorem}[The De Moivre-Laplace Theorem] \label{thm:de-moivre-laplace}
	Let $X_{1}, X_{2}, \ldots$ be i.i.d. with $P\left(X_{1}=1\right)=P\left(X_{1}=-1\right)=1 / 2$ and let $S_{n}=X_{1}+\cdots+X_{n}$. If $a<b$, then as $m \rightarrow \infty$
	\begin{equation}
		P\left(a \leq S_{m} / \sqrt{m} \leq b\right) \rightarrow \int_{a}^{b}(2 \pi)^{-1 / 2} e^{-x^{2} / 2} \dif x.
	\end{equation}
\end{theorem}

\begin{proof}
	If $n$ and $k$ and integers
	\begin{equation*}
		P\left(S_{2 n}=2 k\right)=\left(\begin{array}{c}
				2 n \\
				n+k
			\end{array}\right) 2^{-2 n}
	\end{equation*}
	By lemma \ref{lem:stirling}, we have
	\begin{equation*}
		\begin{aligned}
			\left(\begin{array}{c}
					      2 n \\
					      n+k
				      \end{array}\right) & =\frac{(2 n) !}{(n+k) !(n-k) !}                                                                                           \\
			                       & \sim \frac{(2 n)^{2 n}}{(n+k)^{n+k}(n-k)^{n-k}} \cdot \frac{(2 \pi(2 n))^{1 / 2}}{(2 \pi(n+k))^{1 / 2}(2 \pi(n-k))^{1 / 2}}
		\end{aligned}
	\end{equation*}
	Hence,
	\begin{equation*}
		\begin{aligned}
			P\left(S_{2 n}=2 k\right) & =
			\left(\begin{array}{c}
					      2 n \\
					      n+k
				      \end{array}\right) 2^{-2 n}                                                                                                             \\
			                          & \sim\left(1+\frac{k}{n}\right)^{-n-k} \cdot\left(1-\frac{k}{n}\right)^{-n+k}                                      \\
			                          & \cdot(\pi n)^{-1 / 2} \cdot\left(1+\frac{k}{n}\right)^{-1 / 2} \cdot\left(1-\frac{k}{n}\right)^{-1 / 2}           \\
			                          & =\left(1-\frac{k^{2}}{n^{2}}\right)^{-n} \cdot\left(1+\frac{k}{n}\right)^{-k} \cdot\left(1-\frac{k}{n}\right)^{k} \\
			                          & \cdot(\pi n)^{-1 / 2} \cdot\left(1+\frac{k}{n}\right)^{-1 / 2} \cdot\left(1-\frac{k}{n}\right)^{-1 / 2}           \\
		\end{aligned}
	\end{equation*}
	Let $2k=x\sqrt{2n}$, i.e., $k=x\sqrt{\frac{n}{2}}$. By lemma \ref{lem:exp}, we have
	\begin{equation*}
		\begin{aligned}
			\left(1-\frac{k^{2}}{n^{2}}\right)^{-n} & =\left(1-x^{2} / 2 n\right)^{-n} \rightarrow e^{x^{2} / 2}       \\
			\left(1+\frac{k}{n}\right)^{-k}         & =(1+x / \sqrt{2 n})^{-x \sqrt{n / 2}} \rightarrow e^{-x^{2} / 2} \\
			\left(1-\frac{k}{n}\right)^{k}          & =(1-x / \sqrt{2 n})^{x \sqrt{n / 2}} \rightarrow e^{-x^{2} / 2}
		\end{aligned}
	\end{equation*}
	For this choice of $k$, $k/n \rightarrow 0$, so
	\begin{equation*}
		\left(1+\frac{k}{n}\right)^{-1 / 2} \cdot\left(1-\frac{k}{n}\right)^{-1 / 2} \rightarrow 1.
	\end{equation*}
	Putting things together, we have
	\begin{equation*}
		P\left(S_{2 n}=2 k\right) \sim (\pi n)^{-1 / 2} e^{-x^{2} / 2}, \text{ as } \frac{2k}{\sqrt{2n}} \rightarrow x.
	\end{equation*}
	Therefore,
	\begin{equation*}
		P\left( a\sqrt{2n} \leq S_{2 n} \leq b\sqrt{2 n} \right) = \sum_{m \in \left[a\sqrt{2 n},b\sqrt{2 n}\right] \cap 2\mathbb{Z}} P\left(S_{2 n}=m\right)
	\end{equation*}
	Let $m=x\sqrt{2 n}$, we have that this is
	\begin{equation*}
		\approx \sum_{x \in \left[a,b\right] \cap \left(2\mathbb{Z} / \sqrt{2 n}\right)}(2 \pi)^{-1 / 2} e^{-x^{2} / 2}\cdot(2/n)^{1/2}
	\end{equation*}
	where $2\mathbb{Z} / \sqrt{2 n} = \left\{2z/\sqrt{2n} : z\in\mathbb{Z}\right\}$. As $n\rightarrow\infty$, the sum just shown is
	\begin{equation*}
		\approx \int_{a}^{b}(2 \pi)^{-1 / 2} e^{-x^{2} / 2} \dif x.
	\end{equation*}
	To remove the restriction to even integers, observe $S_{2 n +1}=S_{2 n} \pm 1$.\\
	Let $m=2n$, as $m\rightarrow\infty$,
	\begin{equation*}
		P\left(a \leq S_{m} / \sqrt{m} \leq b\right) \rightarrow \int_{a}^{b}(2 \pi)^{-1 / 2} e^{-x^{2} / 2} \dif x.
	\end{equation*}
\end{proof}

\subsection{Central Limit Theorem}

\begin{theorem}[Classic Central Limit Theorem]
	\label{thm:classic-central-limit-theorem}
	Let $X_1,X_2,\ldots$ be i.i.d. with $\bbE X_{i}=\mu$, $\Var(X_{i})=\sigma^2<\infty$. Let $\bar{X}=\frac{1}{n}\sum_{i=1}^{n}X_{i}$, then
	\begin{equation}
		n^{1/2}\frac{\bar{X}-\mu}{\sigma}\stackrel{d}{\rightarrow}\mcN(0,1).
	\end{equation}
\end{theorem}

\begin{proof}

\end{proof}

\begin{theorem}[The Linderberg-Feller Central Limit Theorem]
	For each $n$, let $X_{n,m},1\leq m\leq n$, be independent random variables with $EX_{n,m}=0$. If
	\begin{enumerate}
		\item $\sum_{m=1}^{n}EX_{n,m}^{2} \rightarrow \sigma^{2}>0$.
		\item $\forall\epsilon>0,\lim_{n\rightarrow\infty}\sum_{m=1}^{n}E\left(\left|X_{n,m}\right|^{2};\left|X_{n,m}\right|>\epsilon\right)=0$
	\end{enumerate}
	Then $S_{n}=X_{n,1}+\cdots+X_{n,n}\stackrel{d}{\rightarrow}\sigma\chi$ as $n\rightarrow\infty$.
\end{theorem}

\subsection{Berry-Esseen Theorem}

\begin{theorem}[Berry-Esseen Theorem]
	Let $X_{1},X_{2},\ldots,X_{n}$ be i.i.d. with distribution $F$ , which has a mean $\mu$ and a finite third moment $\sigma^{3}$, then there exists a constant $C$ (independent of $F$),
	\begin{equation}
		\left|G_{n}(x)-\Phi(x)\right|\leq\frac{C}{\sqrt{n}}\frac{E\left|X_{1}-\mu\right|^{3}}{\sigma^{3}},\quad\forall x.
	\end{equation}
\end{theorem}

\begin{corollary}
	Under the assumptions of Theorem 51,
	$$
		G_{n}(x) \rightarrow \Phi(x) \text { as } n \rightarrow \infty
	$$
	for any sequence $F_{n}$ with mean $\xi_{n}$ and variance $\sigma_{n}^{2}$ for which
	$$
		\frac{E_{n}\left|X_{1}-\xi_{n}\right|^{3}}{\sigma_{n}^{3}}=o(\sqrt{n})
	$$
	and thus in particular if $(72)$ is bounded. Here $E_{n}$ denotes the expectation under $F_{n}$.
\end{corollary}

\section{Central Limit Theorem for independent non-identical Random Variables}

\begin{theorem}[The Liapounov Central Limit Theorem]

\end{theorem}

\begin{theorem}
	Let $Y_1,Y_2,\ldots$ be i.i.d. with $\bbE\left(Y_{i}\right)=0$, $\Var\left(Y_{i}\right)=\sigma^2>$ 0, and $\bbE\left|Y_{i}^3\right|=\gamma<\infty$. If
	\begin{equation}
		\label{eq:sufficient-condition-clt-independent-non-identical-random-variables}
		\left(\sum_{i=1}^n\left|d_{ni}\right|^3\right)^2=o\left(\sum_{i=1}^{n}d_{ni}^2\right)^3,
	\end{equation}
	then
	\begin{equation*}
		\frac{\sum_{i=1}^{n}d_{ni}Y_{i}}{\sigma\sqrt{\sum_{i=1}^{n}d_{n }^2}}\stackrel{d}{\rightarrow}N(0,1).
	\end{equation*}
\end{theorem}

\begin{corollary}
	The sufficient condition \eqref{eq:sufficient-condition-clt-independent-non-identical-random-variables} is euivalent to
	\begin{equation}
		\max_{i=1,\ldots,n}\left(d_{ni}^2\right)=o\left(\sum d_{ni}^2\right).
	\end{equation}
\end{corollary}

\section{Central Limit Theorem for Dependent Random Variables}
