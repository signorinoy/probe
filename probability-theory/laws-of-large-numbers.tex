\chapter{Law of Large Numbers}

\begin{introduction}
    \item Weak Law of Large Numbers
    \item Strong Law of Large Numbers
    \item Uniform Law of Large Numbers
\end{introduction}

\section{Weak Law of Large Numbers}

\begin{lemma}{}{}
    If $p>0$ and $E\left|Z_{n}\right|^{p}\rightarrow 0$, then
    \begin{equation}
        Z_{n}\stackrel{d}{\rightarrow}0.
    \end{equation}
\end{lemma}

\begin{proof}
    
\end{proof}

\begin{theorem}{Weak Law of Large Numbers with Finite Variances}{}
    Let $X_1,X_2,\ldots$ be i.i.d. random variables with $EX_i=\mu$ and $\text{var}(X_i)\leq C<\infty$. Suppose $S_n=X_1+X_2+\ldots+X_n$, then
    \begin{equation}
        S_n/n\stackrel{L^2}{\rightarrow}\mu,\quad S_n/n\stackrel{p}{\rightarrow}\mu.
    \end{equation}
\end{theorem}

\begin{proof}
    
\end{proof}

\begin{theorem}{Weak Law of Large Numbers without i.i.d.}{}
    Let $X_1,X_2,\ldots$ be random variables, Suppose $S_n=X_1+X_2+\ldots+X_n$, $\mu_n=ES_n$, $\sigma_n^2=\text{var}(S_n)$, if $\sigma_n^2/b_n^2\rightarrow 0$, then
    \begin{equation}
        \frac{S_n-\mu_n}{b_n}\stackrel{p}{\rightarrow}0.
    \end{equation}
\end{theorem}

\begin{proof}
    
\end{proof}

\begin{theorem}{Weak Law of Large Numbers for Triangular Arrays}{}
    
\end{theorem}

\begin{proof}
    
\end{proof}

\begin{theorem}{Weak Law of Large Numbers by Feller}{}
    Let $X_1,X_2,\ldots$ be i.i.d. random variables with
    \begin{equation}
        \lim_{x\rightarrow 0}xP(|X_i|>x)=0.
    \end{equation}
    Suppose $S_n=X_1+X_2+\ldots+X_n$, $\mu_n=E\left(X_1I_{(|X_1<n|)}\right)$, then
    \begin{equation}
        S_n/n-\mu_n\stackrel{p}{\rightarrow}0.
    \end{equation}
\end{theorem}

\begin{proof}
    
\end{proof}

\begin{theorem}{Weak Law of Large Numbers}{}
    Let $X_1,X_2,\ldots$ be i.i.d. random variables with $E|X_i|<\infty$. Suppose $S_n=X_1+X_2+\ldots+X_n$, $\mu=EX_i$, then
    \begin{equation}
        S_n/n\stackrel{p}{\rightarrow}\mu.
    \end{equation}
\end{theorem}

\begin{proof}
    
\end{proof}

\begin{note}
    $E|X_i|=\infty$
\end{note}

\section{Strong Law of Large Numbers}

\subsection{Borel-Cantelli Lemmas}

\begin{theorem}{Borel-Cantelli Lemma}{}
    If $\sum_{n=1}^{\infty}P\left(A_{n}\right)<\infty$, then
    \begin{equation}
        P\left(A_{n}\text{ i.o. }\right)=0.
    \end{equation}
\end{theorem}

\begin{theorem}{The Second Borel-Cantelli Lemma}{}
    If $\{A_n\}$ are independent with $\sum_{n=1}^{\infty}P\left(A_{n}\right)=\infty$, then,
    \begin{equation}
        P\left(A_{n}\text{ i.o. }\right)=1.
    \end{equation}
\end{theorem}

\subsection{Strong Law of Large Numbers}

\begin{theorem}{Strong Law of Large Numbers}{SLLN}
    Let $X_1,X_2,\ldots$ be i.i.d. random variables with $E|X_i|<\infty$. Suppose $S_n=X_1+X_2+\ldots+X_n$, $\mu=EX_i$, then
    \begin{equation}
        S_n/n\stackrel{a.s.}{\rightarrow}\mu.
    \end{equation}
\end{theorem}

\section{Uniform Law of Large Numbers}